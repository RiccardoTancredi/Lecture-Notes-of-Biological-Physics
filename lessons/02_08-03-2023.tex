\documentclass[../main/main.tex]{subfiles}

\begin{document}
\section{Cell physiology}

\newdate{date}{08}{03}{2023}

\marginpar{ \textbf{Lecture 3.} \\  \displaydate{date}. \\ Compiled:  \today.}

To keep themselves organized, the first thing cells are able to do is to create a clear distinction between the inside and outside world, so between their inner and outer anatomy. Even though there are more than 200 specialized cells in the human body, they all share some common features. Thinking of a cell as separate entities, they are still organism that, in order to work properly, have to take and store energy. A fraction of this energy is used for mechanical activity or for metabolismo - a collection of processes that allows cells to maintein their homeostasis. Energy is further engaged in protein assembly, essential molecules of cells internal structure, or to allow mitosis (duplicating) processes.
The inner homeostasis is kept not only by maintaining a fixed interior volume, but also by balancing the concentration of ions there present. Nerves and muscles use this mechanism also to send and receive signals, so to communicate. Communication is of primary importance in order to sense the environmental conditions, so to regulate their interior composition. The presence of needs and food supply or enemy cells has to trigger cells correct response. As part of a feedback loop, cells can also sense their internal conditions and stop particular internal molecular machines.
A cell can even destroy itself, in a process called \emph{apoptosis}, in which a specific \emph{protease} is activated because of cell stress, or because of signals from other cells (\emph{cell communication}). 

\end{document}